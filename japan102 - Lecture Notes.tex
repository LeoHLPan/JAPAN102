\documentclass{article}

\usepackage[utf8]{inputenc}
\usepackage{fullpage}
\usepackage{times}
\usepackage{tcolorbox}
\usepackage{enumitem}
\usepackage{multicol}
\usepackage{bookmark}
\usepackage{CJKutf8}
\usepackage[normalem]{ulem}
\usepackage{qtree}
\usepackage{tree-dvips}

\bookmarksetup{
  numbered, 
  open,
}

\setlist{itemsep=2pt}
\renewcommand{\thesection}{Lecture \arabic{section}}
\renewcommand{\thesubsection}{\arabic{section}.\arabic{subsection}}
%\renewcommand{\thesubsubsection}{\indent\arabic{section}.\arabic{subsection}.\arabic{subsubsection}}
\setcounter{section}{0}

\begin{document}
\begin{CJK}{UTF8}{min}

\noindent
{JAPAN 102 \hfill Hao Pan}\\
{Shimoda, Fumie}\\
{Fall 2018}

%%%%%%%%%%%%%%%%%%%%%%%%%%%%%%%%%%%%%%%%%%%%%%%%%%%%%%%%%%%

\begin{center}
\section{}
\noindent
{\hfill 06/09/2018 [Th]}
\end{center}

\subsection{Review}

\uline{v.te-form}
\begin{itemize}
\item Polite requests: ください.
\item Giving permission: もいいです.
\item Forbidden: はいけもせん.
\end{itemize}


\subsection{Action in Progress}
[v.te-form]います。

\begin{itemize}
\item \textbf{Ex}. \uline{たべて}います。 | \textit{I am eating.}
\end{itemize}


\subsection{Occupation or Habit}
[occupation]をしています。

\begin{itemize}
\item \textbf{Ex}. にほんごのせんせいを\uline{しています}。 | \textit{I \uline{am a} Japanese teacher.}
\item \textbf{Ex}. にほんごを\uline{べんきょしています}。 | \textit{I am \uline{studying} Japanese.}
\end{itemize}


\subsection{Changes from One State to Another}
[v.te-form]います。

\bigskip

\textbf{\uline{Vocabulary}}\\

\begin{tabular}{ | c | c | c | }
\hline
Word & Te-form Sentence & Translation\\
\hline
ふとる & ふとっています & I am overweight\\
やせる & やせています & I am thin\\
すわる & すわっています & I am sitting\\
\hline
\end{tabular}

\bigskip

\begin{tabular}{ | c | c | c | c | c | }
\hline
Word & Translation & Group & Connector & Negative\\
\hline
\hline
すむ & To live (in a place) & 1 & - & -\\
つとめる & To work for & 2 & _に_ & _いません\\
けっこんする & To get married & 2 & - & -\\
\hline
しる & To know & 1 & _を_ & しりません\\
\hline
もつ & To own/carry & 1 & _を_ & もちません\\
\hline
\end{tabular}


\subsection{Putting on Clothes}

\begin{tabular}{ | c | c | c | }
\hline
Word & Usage & Group\\
\hline
きる & Above waist & 2\\
はく & Below waist & 1\\
かぶる & Hat & 1\\
かける & Glasses & 2\\
する & Accessories & 3\\
\hline
\end{tabular}
\begin{itemize}
\item \textbf{Usage}: [N]を[v.te-form]。
\end{itemize}


\subsection{Physical Attributes}

せ | \textit{height}

\begin{itemize}
\item \textbf{Ex}. やまださんはせ\textbf{が}\uline{たかい}です。 | \textit{Mr. Yamada is \uline{tall}.}
\begin{itemize}
\item ひくい | \textit{short}
\end{itemize}
\item かみ\textbf{が}\uline{ながい}です。 | \textit{long hair}
\begin{itemize}
\item みじかい | \textit{short hair}
\end{itemize}
\item め\textbf{が}\uline{おおきい}です。 | \textit{big eyes}
\begin{itemize}
\item ちいさい | \textit{small eyes}
\end{itemize}
\item くち | \textit{mouth}
\end{itemize}
















\end{CJK}
\end{document}
